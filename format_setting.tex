%% format setting for schedule of cscm2017
\renewcommand{\normalsize}{\zihao{-4}}

\setmainfont{Times New Roman}

\geometry{a4paper,left=2.5cm, right=2.5cm, top=2.5cm, bottom=2.5cm}

\CTEXsetup[name={,}, number={\chinese{chapter}}]{chapter}
\CTEXsetup[name={,}, number={}, format+={\centering}]{section}
\CTEXsetup[name={,}, number={}, format+={}]{subsection}

\makeatletter
\def\cleardoublepage{\clearpage\if@twoside \ifodd\c@page\else
  \hbox{}
  \vspace*{\fill}
  \begin{center}
    \Large\quad
  \end{center}
  \vspace{\fill}
  \thispagestyle{empty}
  \newpage
  \if@twocolumn\hbox{}\newpage\fi\fi\fi}
\makeatother
\pagestyle{fancy}

\renewcommand{\headrulewidth}{0.4pt}
\fancypagestyle{plain}{\thispagestyle{fancy}}

\def\hascover#1{}
\def\hasmap#1{}

%% for schedule table
\newlength{\npuWidth}
\newlength{\npuTenHeight}
\newlength{\npuTempLen}
\setlength{\npuWidth}{3.4cm}
\setlength{\npuTenHeight}{8pt}
\def\setNpuLength#1#2{\setlength{#1}{#2}\addtolength{#1}{-4\fboxsep}\addtolength{#1}{-2\fboxrule}}
\def\mybox[#1][#2]#3{\fbox{\colorbox{#2}{\parbox[t][#1][c]{\npuWidth}{\begin{center}#3\end{center}}}}}
\def\eventbox[#1][#2]#3{%
  \setNpuLength{\npuTempLen}{#1\npuTenHeight}\mybox[\npuTempLen][#2]{#3}\\
}
\newenvironment{schedule}{%
  \zihao{-4}
  \let\oldtabcolsep=\tabcolsep
  \let\oldextrarowheight=\extrarowheight
  \let\oldfboxrule=\fboxrule
  \let\oldfboxsep=\fboxsep
  \fboxrule=0.5pt
  \fboxsep=1pt
  \tabcolsep=0pt
  \extrarowheight=-20pt
  \begin{center}
}{%
  \end{center}
  \let\tabcolsep=\oldtabcolsep
  \let\extrarowheight=\oldextrarowheight
  \let\fboxrule=\oldfboxrule
  \let\fboxsep=\oldfboxsep
  \normalsize
}


%% for plenary talk
\def\plenaryItem#1#2#3#4#5#6{%
  \item {\bf \nputitle{#1}{#6}} \\%
  #2\index{#2} \quad #3 ({\it #4}) \\%
  #5%
}
\def\nputitlenew#1#2{#1}
\def\plenary#1#2#3#4#5#6#7{%
  报告人: #2\index{#2} \quad #3 ({\it #4})\\
  题目: \nputitle{#1}{#6} \\\cnhosts: #7 }

\def\cnorganizors{组织者}
\def\cnhosts{主持人}

%% for short information
\newcolumntype{f}{>{\columncolor{gray!30}\bf\centering}p{50pt}}
\newcolumntype{g}{>{\columncolor{gray!15}}X}
\newcolumntype{h}{>{\columncolor{gray!30}}X}
\settowidth\npuTempLen{00:00--00:00}
\long\def\session#1#2{\clearpage\section{\bf{ #1 | {\rm #2}}}}
\def\specialtable#1#2{%
  报告人:  & #1 \\%
  时间:    & #2 \\%
  \hline%
}

% #1 专题名称 #2 会议室 #3 组织者 #4 报告人列表  #5 cnhosts 或者 cnorganizors  #6 链接 label
\long\def\specialtalk#1#2#3#4#5#6{% 
  % \vskip30pt
  \noindent
  \parbox[b]{\textwidth}{%
    \parindent = 0pt
    {\hypertarget{#6}{\color{gray}\rule{\textwidth}{0.8mm}}}\\
    {\bf {#1}} \hfill \colorbox{gray!30}{#2}
    \begin{tabularx}{\textwidth}{fghgh} 
      #5:  & \multicolumn{4}{l}{#3} \\ 
      \hline
      #4
    \end{tabularx}%
  } %
  \vskip30pt%
}

\newlength{\npunewlen}
\settowidth\npunewlen{10001\quad asprof}


%% for detail information of reports
\def\npuauthor#1#2#3{\hyperref[#1]{#2}\index{#3@#2}}
\def\npuauthornew#1#2#3#4#5{\makecell[tl]{\hyperref[#1]{#2}\index{#5@#2}\\{{#3\quad #4}}}}

\makeatletter
\newcommand{\labeltext}[2]{%
  \@bsphack
  \csname phantomsection\endcsname % in case hyperref is used
  \def\@currentlabel{#1}{\label{#2}}%
  \@esphack
}
\makeatother
% \def\nputopic#1{\vskip0pt\noindent\rule{0.3\textwidth}{0.6mm}\\ {#1}}
\def\sectocont#1{\addcontentsline{toc}{section}{#1}}
\def\subsectocont#1{\addcontentsline{toc}{subsection}{#1}}
\def\nputopic#1{
  % \vskip0pt
  \noindent\rule{0.3\textwidth}{0.6mm}\\%
  {\bf #1%\sectocont{\qquad\quad#1}
  \\}%
}

\def\npuleader#1#2{%
  \mbox{%
    {%
      {\bf #1:}\hskip10pt%
      \parbox[t]{\textwidth}{#2}%
    }%
  }%
}
\def\npuwhat#1#2{{\bf #1:}\hskip10pt #2}
\def\nputime#1#2{\npuwhat{时间}{#1}\qquad\npuwhat{会场}{#2}\\}
\def\nputitle#1#2{\hyperlink{#2}{#1}}

\def\npureportlist#1{ %
  \vskip10pt\noindent
  \let\odlarraystretch\arraystretch
  \renewcommand\arraystretch{1.2}
  \rowcolors{2}{gray!15}{white}
  \begin{tabularx}{\linewidth}{>{\centering}p{137pt}|X}
    \rowcolor{gray!15}\hline
    {\bf 报告人} & {\bf 题目 }\\\hline #1 \hline
  \end{tabularx}
  \let\arraystretch\odlarraystretch
}

\def\npuauthorschool#1#2#3{%
  #1\index{#3@#1}{\small({\it #2})}%
}
\def\npuauthorschoolc#1#2#3#4#5{%
  #1\index{#5@#1}{\labeltext{#1}{#3}\hfill\small(#4)\\{\it #2\hfill}} %
}
\def\npureport#1#2#3#4#5#6{\npuauthorschoolc{#3}{#4}{#2}{#5}{#6} & {#1}\\}

%% for abstract list
\def\npumail#1{\ifthenelse{\isempty{#1}}%
  {}% if #1 is empty
  {Email: \href{mailto:#1}{#1}}% if #1 is not empty
}

\def\npufootauthor#1#2#3#4{#4\index{#1}{\footnote{#1, #2. \npumail{#3}}}}
\def\npufootstart#1#2#3{#2\index{#3@#1}\stepcounter{footnote}\footnotemark[\thefootnote]}
\def\npufootend#1#2#3{\footnotetext[\thefootnote]{#1, #2. \npumail{#3}}}
\long\def\npuabstract#1#2#3#4#5{%
  \vbox{%
    \hrule width 0pt height 5pt
    \noindent
    %\labeltext{#2}{#2}
    \begin{center}
    \bf\hypertarget{#2}{#1}
    \end{center}
    \noindent
    \hfil{#3}\hfil
    
    {\bf Abstract: }
    #4

    \hrule width 0pt height 5pt
     %
  }%
  #5
}

\def\npufootstartold#1#2{#2\index{#1}\stepcounter{footnote}\footnotemark[\thefootnote]}

